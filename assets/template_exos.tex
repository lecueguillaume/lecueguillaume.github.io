\documentclass[11pt]{article}
\usepackage[utf8]{inputenc}

\usepackage{amsmath, amsthm, amsfonts, amssymb}
\usepackage{dsfont,url,hyperref} % utilisation : {\mathds N} (au lieu de mathbb)
\usepackage{mathrsfs} % utilisation : {\mathscr B} (plus joli que mathcal qu'il remplace)
\usepackage{multirow}
\usepackage{mathtools}
\usepackage{enumerate}
\usepackage{numprint}

%\usepackage{epsfig,graphics}

%\usepackage{caption}
%\captionsetup[figure]{labelformat=empty}

%\usepackage{ccaption}

%\usepackage{verbatim} % utile notamment pour l'environnement "comment"

\usepackage[T1]{fontenc}
%\usepackage[latin1]{inputenc}
\usepackage[french]{babel}

\usepackage{xcolor}
\usepackage{fancybox}
\definecolor{mongris}{gray}{0.9}

%%%%%%%%%%%%%%%% mise en page
\setlength{\topmargin}{-1.7cm} \setlength{\oddsidemargin}{0cm}
\setlength{\evensidemargin}{0cm} \setlength{\textwidth}{15.5truecm}
\setlength{\textheight}{25truecm}

\def\ds{\displaystyle}

\newcommand{\C}{{\mathbb C}}
\newcommand{\E}{{\mathbb E}}
\newcommand{\N}{{\mathbb N}}
\renewcommand{\P}{{\mathbb P}}
\renewcommand{\Pr}{{\mathbb P}}
\newcommand{\R}{{\mathbb R}}
\newcommand{\Z}{{\mathbb Z}}

\newcommand{\ii}{{\textup{i}}}
\newcommand{\ee}{{\textup{e}}}

\newcommand{\boF}{{\mathscr F}}
\newcommand{\boG}{{\mathscr G}}
\newcommand{\boB}{{\mathscr B}}

\newcommand{\e}{\operatorname{e}} % pour faire des exponentielles avec un "e" droit mais surtour pour que LateX le gere comme un vrai operateur au niveau des espaces

\renewcommand{\d}{\mathrm{d}}


\newcommand{\cv}[1][n]{\enskip\mathop{\longrightarrow}^{}_{#1 \to \infty}\enskip}
\newcommand{\cvloi}[1][n]{\enskip\mathop{\longrightarrow}^{(loi)}_{#1 \to \infty}\enskip}
\newcommand{\cvps}[1][n]{\enskip\mathop{\longrightarrow}^{p.s.}_{#1 \to \infty}\enskip}
\newcommand{\cvproba}[1][n]{\enskip\mathop{\longrightarrow}^{\P}_{#1 \to \infty}\enskip}


\DeclareMathOperator{\Pto}{\stackrel{\P}{\longrightarrow}}
\DeclareMathOperator{\Lto}{\stackrel{\scriptscriptstyle{\textup{loi}}}{\longrightarrow}}
\DeclareMathOperator{\PSto}{\stackrel{p.s.}{\longrightarrow}}

\DeclareMathOperator{\ind}{{\mathds 1}}
\DeclareMathOperator{\Var}{\mathop{\rm Var}\nolimits}

\newcounter{definition}[section]
\theoremstyle{definition} 
\newtheorem{exo}{Exercice}
\newtheorem{exercice}{Exercice}
\newtheorem{exercice-star}{Exercice}
\newtheorem*{solution}{Solution}
\newtheorem*{solution-cor}{Solution}

\newif\ifhidesolutions
%\hidesolutionstrue %decommenter pour cacher les SOLUTIONS

\ifhidesolutions
\usepackage{environ}
\NewEnviron{hide}{}
\let\solution\hide
\let\endsolution\endhide
\fi

 % barre pour separer les exercices 
\newcommand{\barre}{
\begin{center}
\rule{.3\linewidth}{1pt}
\end{center}
\medskip}

\parindent=0pt






%%%%%%%%%
\begin{document}

\textsc{\'Ecole Polytechnique}\hskip 3.2cm {\bf MAP 361} \hskip 4.5cm
2019/2020 \par \hrule

\par\vspace{0.5cm}

\begin{center} {\bf \Large PC~0: Organisation}\end{center}

\barre %%%%%%%%%%%%%%%%%%%%%%%%%%%%%%%%%%%%%%%%%%%%%%%%%%%

La PC0 est une PC d'organisation pour les 9 PCs suivantes. N'h{\'e}sitez pas {\`a} me faire des retours sur les probl{\`e}mes que vous avez identifi{\'e}s lors de cette scéance. L'exercice suivant est donn{\'e} {\`a} dans ce cadre.

\barre %%%%%%%%%%%%%%%%%%%%%%%%%%%%%%%%%%%%%%%%%%%%%%%%%%%




\begin{exercice}[\textsc{Conditionnement}]
L'exercice suivant est tr\`es classique et a de nombreuses variantes. Il
illustre l'utilit\'e d'une formulation math\'ematique rigoureuse pour \'eviter
des pi\`eges et des paradoxes dus \`a des raisonnements sp\'ecieux.\\
Une famille a deux enfants. On suppose les 4 configurations
$(\omega_1,\omega_2)$ avec $\omega_i$ le sexe du $i$\`eme enfant \'equiprobables. 
\begin{enumerate}
%\item Quelle est la probabilit\'e pour que les deux enfants soient des filles sachant que le plus jeune enfant est une fille?
%\item Quelle est la probabilit\'e pour que les deux enfants soient des filles sachant que l'enfant plus âg\'e est une fille?
%\item Quelle est la probabilit\'e pour que les deux enfants soient des filles sachant que l'un des enfants  est une fille?
\item Montrer que la probabilit\'e pour que les deux enfants soient des filles sachant que le plus jeune enfant est une fille vaut $\frac{1}{2}$.
\item Montrer que la probabilit\'e pour que les deux enfants soient des filles sachant que l'enfant plus âg\'e est une fille vaut $\frac{1}{2}$.
\item Montrer que la probabilit\'e pour que les deux enfants soient des filles sachant que l'un des enfants est une fille vaut $\frac{1}{3}$.
\end{enumerate}
\end{exercice}

\medskip

\begin{solution}
On note l'espace fondamental $\Omega$. Ici il vaut
\begin{equation*}
\Omega = \left\{(F,F), (F,G), (G,F), (G,G)\right\}.
\end{equation*}On munit cet espace de la tribu $\mathcal A$ faites de tous les sous-ensembles de $\Omega$:
\begin{equation*}
\mathcal A=\left\{\emptyset, \{(F,F)\}, \cdots, \Omega\right\}.
\end{equation*}Cet ensemble est de cardinal $2^4=16$. Finalement, on munit l'espace mesurable $(\Omega, \mathcal A)$ d'une mesure de probabilit{\'e} $\mathbb{P}$. L'énoncé nous dit que les $4$ événements $\{(F,F)\}, \{(F,G)\}, \{(G,F)\}$ et $\{(G,G)\}$ sont équiprobables; on pose donc
\begin{equation*}
\mathbb{P}(\{(F,F)\})=\mathbb{P}( \{(F,G)\})=\mathbb{P}(\{(G,F)\})=\mathbb{P}(\{(G,G)\})=\frac{1}{4}
\end{equation*}et par la propri{\'e}t{\'e} de $\sigma$-additivit{\'e} de $\mathbb P$, on étends la définition de $\mathbb P$ à toute la tribu $\mathcal A$.
\end{solution}

Voil{\'a} pour le cadre. Pour les probl{\`e}mes de conditionnement, il est important de passer un peu de temps {\`a} bien d{\'e}finir ce cadre probabiliste pour {\'e}viter des erreurs. On passe, maintenant, aux réponse aux questions. 

La seule connaissance {\`a} avoir pour r{\'e}soudre ce probl{\`e}me est la d{\'e}finition du conditionnement : si $A$ et $B$ sont deux {\'e}v{\'e}nements (c{\`a}d deux {\'e}l{\'e}ments de $\mathcal A$) alors la probabilité que $A$ est lieu sachant $B$ est not{\'e} et d{\'e}fini par
\begin{equation*}
\P(A|B)=\frac{\P(A\cap B)}{\P(B)}.
\end{equation*}

\begin{enumerate}
\item
 L'\'ev\'enement que le plus jeune enfant est une fille correspond à l'ensemble $A\coloneqq\{(F,F),(G,F)\}$ qui se produit avec probabilit\'e $\P(A)=|A| / |\Omega|=1/2$. Pour la probabilit\'e que les deux
enfants soient des filles sachant que le plus jeune enfant est une fille on obtient alors
\[\P(\{(F,F)\} \mid A)=\frac{\P(\{(F,F)\} \cap A)}{\P(A)}=\frac{\P(\{(F,F)\} )}{\P(A)}=\frac{1/4}{1/2}=\frac12.\]
\item Par le même raisonnement, en notant  $B\coloneqq\{(F,F),(F,G)\}$ l'\'ev\'enement que l'enfant plus âg\'e est une fille avec  $\P(B)=|B| / |\Omega|=1/2$, on obtient 
\[\P(\{(F,F)\} \mid B)=\frac{\P(\{(F,F)\} \cap B)}{\P(B)}=\frac{\P(\{(F,F)\} )}{\P(B)}=\frac{1/4}{1/2}=\frac12.\]
\item Notons  $C\coloneqq\{(F,F),(F,G),(G,F)\}$ l'\'ev\'enement que l'un des enfants  est une fille. On a  $\P(C)=|C| / |\Omega|=3/4$ et donc
\[\P(\{(F,F)\} \mid C)= \frac{\P(\{(F,F)\} )}{\P(C)}=\frac{1/4}{3/4}=\frac13.\]
\end{enumerate}











\end{document}


\begin{exercice}[\textsc{}]
\end{exercice}

\begin{solution}
\end{solution}


\begin{exercice}[\textsc{}]
\end{exercice}

\begin{solution}
\end{solution}